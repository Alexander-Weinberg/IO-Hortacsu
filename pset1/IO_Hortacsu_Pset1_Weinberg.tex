%% LyX 2.3.6.2 created this file.  For more info, see http://www.lyx.org/.
%% Do not edit unless you really know what you are doing.
\documentclass[english]{article}
\usepackage[T1]{fontenc}
\usepackage[latin9]{inputenc}
\usepackage{geometry}
\geometry{verbose,tmargin=1in,bmargin=1in,lmargin=1in,rmargin=1in}
\usepackage{amsmath}
\usepackage{amssymb}

\makeatletter
%%%%%%%%%%%%%%%%%%%%%%%%%%%%%% User specified LaTeX commands.
\usepackage{mdframed}
\usepackage[dvipsnames]{xcolor}


\newenvironment{problem}[2][Problem]
    { \begin{mdframed}[backgroundcolor=gray!20] \textbf{#1 #2} \\}
    {  \end{mdframed}}


\newcommand{\R}{\mathbb{R}}
\newcommand{\E}{\mathbb{E}}
\newcommand{\next}{^{\prime}}

\makeatother

\usepackage{babel}
\begin{document}
\title{\textbf{Industrial Organization-II}: Problem Set 1}
\author{Monica Barbosa and Alex Weinberg}
\date{\today}
\maketitle

\section{Model}

\begin{problem}{1}

Let $X_{i}$ with $i=1, \ldots, N$ be a sequence of independent type 1 extreme value random variables with location parameter $\mu_{i}$ and scale parameter $\sigma>0\left(\mathrm{~T} 1 \mathrm{EV}\left(\mu_{i}, \sigma\right)\right)$. 

The c.d.f is given by:

$$ \operatorname{Pr}\left\{X_{i} \leq x \mid \mu_{i}, \sigma\right\}=\exp \left(-\exp \left(-\frac{x-\mu_{i}}{\sigma}\right)\right) $$

Derive the distribution of $Y=\max _{i}\left\{X_{i}\right\}$.
\end{problem}

\paragraph{Solution. }

\begin{align*}
Pr\left(Y\leq y\right) & =Pr\left(\max_{i}X_{i}\leq y\right)\\
 & =\prod_{i=1}^{N}Pr\left(X_{i}\leq y\right)\\
 & =\prod_{i=1}^{N}e^{\left(-e^{\left(-\frac{y-\mu_{i}}{\sigma}\right)}\right)}\\
 & =e^{\left(-e^{\left(-\frac{y-\mu_{1}}{\sigma}\right)}\right)}\times\dots\times e^{\left(-e^{\left(-\frac{y-\mu_{N}}{\sigma}\right)}\right)}\\
 & =e^{\left(-\sum_{i=1}^{N}e^{\left(-\frac{y-\mu_{i}}{\sigma}\right)}\right)}\\
 & =e^{\left(-\sum_{i=1}^{N}e^{\frac{-y}{\sigma}}e^{\frac{\mu_{i}}{\sigma}}\right)}\\
 & =e^{\left(-e^{\frac{-y}{\sigma}}\sum_{i=1}^{N}e^{\frac{\mu_{i}}{\sigma}}\right)}\\
 & =e^{\left(-e^{\frac{-y}{\sigma}}e^{\log\left(\sum_{i=1}^{N}e^{\frac{\mu_{i}}{\sigma}}\right)}\right)}\\
 & =e^{\left(-e^{-\left[\frac{y-\sigma\log\left(\sum_{i=1}^{N}e^{\frac{\mu_{i}}{\sigma}}\right)}{\sigma}\right]}\right)}
\end{align*}

Notice that is this a CDF of a T1EV distribution with location parameter
$\sigma\log\left(\sum_{i=1}^{N}e^{\frac{\mu_{i}}{\sigma}}\right)$
and scale parameter $\sigma$.

\begin{problem}{2}

Let $X$ and $Y$ two independent T1EV random variables with location parameters $\mu_{x}$ and $\mu_{y}$ respectively and common scale parameter $\sigma>0$. Derive the distribution of $X-Y$.
\end{problem}

Suppose 
\begin{align*}
X & \overset{iid}{\sim}T1EV\left(\mu_{x},\sigma\right)\\
Y & \overset{iid}{\sim}T1EV\left(\mu_{y},\sigma\right)
\end{align*}

Denote the CDF of X and Y respectively as $F_{x},F_{y}$. Use $f_{x},f_{y}$
as the densities. Want to find the distribution of the difference.

\begin{problem}{3}
Consider an individual who has to choose one product among $N$ possible alternatives. The utility derived from alternative $j$ is given by:
$$ u_{j}=\mu_{j}+\epsilon_{j} $$
where $\mu_{j}$ is non-random and $\epsilon_{j}$ are independent and identically distributed T1EV(0,1). Derive the probability that alternative $j$ is chosen.
\end{problem}

This answer follows Chapter 3 of Train's textbook. 
\begin{align*}
Pr\left(i\rightarrow j\right) & =Pr\left(u_{j}\geq u_{k}\ \forall k\right)\\
 & =Pr\left(\mu_{j}+\epsilon_{ij}\geq\mu_{k}+\epsilon_{ik}\ \forall k\right)\\
 & =Pr\left(\mu_{j}+\epsilon_{ij}-\mu_{k}\geq\epsilon_{ik}\ \forall k\right)\\
\end{align*}

Suppose now that we fix $\epsilon_{ij}$ then from the assumption
that $\epsilon_{ik}\overset{iid}{\sim}T1EV$ we get
\begin{align*}
Pr\left(i\rightarrow j\right)|\epsilon_{ij} & =\prod_{k\neq j}Pr\left(\mu_{j}+\epsilon_{ij}-\mu_{k}\geq\epsilon_{ik}\right)\\
 & =\prod_{k\neq j}F\left(\mu_{j}+\epsilon_{ij}-\mu_{k}\geq\epsilon_{ik}\right)\\
 & =\prod_{k\neq j}e^{-e^{-\left(\mu_{j}+\epsilon_{ij}-\mu_{k}\right)}}
\end{align*}

Integrating over $\epsilon_{ij}$ and recalling that $f\left(\epsilon_{ij}\right)=e^{-\epsilon_{ij}}e^{-e^{-\epsilon_{ij}}}$
we get 
\begin{align*}
Pr\left(i\rightarrow j\right) & =\int\left[\prod_{k\neq j}e^{-e^{-\left(\mu_{j}+\epsilon_{ij}-\mu_{k}\right)}}\right]e^{-\epsilon_{ij}}e^{-e^{-\epsilon_{ij}}}d\epsilon_{ij}\\
 & =\int\left[\prod_{k\neq j}e^{-e^{-\left(\mu_{j}+\epsilon_{ij}-\mu_{k}\right)}}e^{-e^{-\epsilon_{ij}}}\right]e^{-\epsilon_{ij}}d\epsilon_{ij}\\
 & =\int\left[\prod_{k=1}^{J}e^{-e^{-\left(\mu_{j}+\epsilon_{ij}-\mu_{k}\right)}}\right]e^{-\epsilon_{ij}}d\epsilon_{ij}\\
 & =\int exp\left(\log\left[\prod_{k=1}^{J}e^{-e^{-\left(\mu_{j}+\epsilon_{ij}-\mu_{k}\right)}}\right]\right)e^{-\epsilon_{ij}}d\epsilon_{ij}\\
 & =\int exp\left(\left[-\sum_{k=1}^{J}e^{-\left(\mu_{j}+\epsilon_{ij}-\mu_{k}\right)}\right]\right)e^{-\epsilon_{ij}}d\epsilon_{ij}\\
 & =\int exp\left(\left[-e^{-\epsilon_{ij}}\sum_{k=1}^{J}e^{-\left(\mu_{j}-\mu_{k}\right)}\right]\right)e^{-\epsilon_{ij}}d\epsilon_{ij}\\
\end{align*}

Define $t=e^{-\epsilon_{ij}}$. Notice that as $\epsilon_{ij}\rightarrow\infty$
then $t\rightarrow0$. As $\epsilon_{ij}\rightarrow-\infty$ then
$t\rightarrow\infty$. So we can re-write the integral as
\begin{align*}
Pr\left(i\rightarrow j\right) & =\int exp\left(-t\sum_{k=1}^{J}e^{-\left(\mu_{j}-\mu_{k}\right)}\right)dt\\
 & =\frac{exp\left(-t\sum_{k=1}^{J}e^{-\left(\mu_{j}-\mu_{k}\right)}\right)}{-\sum_{k=1}^{J}e^{-\left(\mu_{j}-\mu_{k}\right)}}|_{0}^{\infty}\\
 & =0-\frac{e^{0}exp\left(\sum_{k=1}^{J}e^{-\left(\mu_{j}-\mu_{k}\right)}\right)}{-\sum_{k=1}^{J}e^{-\left(\mu_{j}-\mu_{k}\right)}}\\
 & =\frac{1}{\sum_{k=1}^{J}e^{-\left(\mu_{j}-\mu_{k}\right)}}\\
 & =\frac{1}{\sum_{k=1}^{J}e^{-\mu_{j}}e^{\mu_{k}}}\\
 & =\frac{e^{\mu_{j}}}{\sum_{k=1}^{J}e^{\mu_{k}}}\\
\end{align*}

\begin{problem}{4}
Consider a market with $J$ products indexed by $j=1, \ldots, J$, an outside good denoted by $j=0$ and a large number of consumers indexed by $i \in \mathcal{I}$ each of whom only buys one of the products. Consumer $i$ 's indirect utility from consuming product $j$ is given by:
$$ \begin{aligned} & u_{i j}=\alpha\left(y_{i}-p_{j}\right)+\epsilon_{i j} \quad \text { for } j=1, \ldots, J \\ & u_{i 0}=\alpha y_{i}+\epsilon_{i 0} \quad \text { for } j=0 \end{aligned} $$
where $p_{j}$ is the price of product $j, y_{i}$ is consumer $i$ 's income and $\epsilon_{i j}$ is an idiosyncratic taste shock that makes products horizontally differentiated.
\end{problem}

\begin{problem}{4.a}
Assume $\epsilon_{i j}$ are i.i.d T1EV(0,1). Denote consumer $i$ 's individual choice probability of selecting product $j$ as $s_{j}(i)$. Derive $s_{j}(i)$ and compute $\frac{\partial s_{j}(i)}{\partial y_{i}}$. Interpret your results.
\end{problem}

\[
\epsilon_{ij}\begin{subarray}{c}
\overset{iid}{\sim}\end{subarray}T1EV\left(0,1\right)
\]

We want to compute $s_{j}\left(i\right)$
\begin{align*}
s_{j}\left(i\right) & =Pr\left(i\rightarrow j\right)\\
 & =Pr\left(u_{ij}\geq u_{ik}\ \forall k\neq j\right)\\
 & =Pr\left(\alpha y_{i}-\alpha p_{j}+\epsilon_{ij}\geq\alpha y_{i}-\alpha p_{k}+\epsilon_{ik}\ \forall k\neq j\right)\\
 & =Pr\left(-\alpha p_{j}+\epsilon_{ij}\geq-\alpha p_{k}+\epsilon_{ik}\ \forall k\neq j\right)\\
 & =Pr\left(-\alpha p_{j}+\epsilon_{ij}+\alpha p_{k}\geq\epsilon_{ik}\ \forall k\neq j\right)\\
 & =\int\left[\prod_{k\neq j}F\left(-\alpha p_{j}+\epsilon_{ij}+\alpha p_{k}\right)\right]f\left(\epsilon_{ij}\right)d\epsilon_{ij}\tag{b/c \ensuremath{\epsilon_{ij}\begin{subarray}{c}
\overset{iid}{\sim}\end{subarray}}T1EV\ensuremath{\left(0,1\right)}}\\
 & =\int\left[\prod_{k\neq j}e^{-e^{-\left(-\alpha p_{j}+\epsilon_{ij}+\alpha p_{k}\right)}}\right]e^{-\epsilon_{ij}}e^{-e^{-\epsilon_{ij}}}d\epsilon_{ij}\\
 & =\frac{e^{-\alpha p_{j}}}{\sum_{k=0}^{J}e^{-\alpha p_{k}}}
\end{align*}

where $p_{0}$ is normalized to $0$.

Clearly income elasticity is zero.
\begin{align*}
\frac{\partial s_{j}\left(i\right)}{\partial y_{i}} & =0\\
\end{align*}

This is sensible in this context because income effects affect all
goods the same. The consumer is required to choose one and only one
option (including the outside option). Changes in her income will
not affect this.

\begin{problem}{4.b}
Assume $\epsilon_{i j}$ are i.i.d T1EV $(0,1)$. Derive $s_{j}$ (the market share of product $j$ ) and compute own and cross-price elasticities. Are the latter reasonable? Explain.
\end{problem}

\paragraph{Own price elasticity.
\begin{align*}
E_{jj} & =\frac{\partial s_{j}}{\partial p_{j}}\frac{p_{j}}{s_{j}}\protect\\
\end{align*}
}

Start with the own-price derivative.
\begin{align*}
\frac{\partial s_{j}}{\partial p_{j}} & =\frac{\partial}{\partial p_{j}}\left[\frac{e^{-\alpha p_{j}}}{\sum_{k=0}^{J}e^{-\alpha p_{k}}}\right]\\
 & =\frac{-\alpha e^{-\alpha p_{j}}\sum_{k=0}^{J}e^{-\alpha p_{k}}-\left(-\alpha\right)e^{-\alpha p_{j}}e^{-\alpha p_{j}}}{\sum_{k=0}^{J}e^{-\alpha p_{k}}\times\sum_{k=0}^{J}e^{-\alpha p_{k}}}\\
 & =\frac{-\alpha e^{-\alpha p_{j}}}{\sum_{k=0}^{J}e^{-\alpha p_{k}}}\frac{\sum_{k=0}^{J}e^{-\alpha p_{k}}-e^{-\alpha p_{j}}}{\sum_{k=0}^{J}e^{-\alpha p_{k}}}\\
 & =-\alpha s_{j}\left(1-s_{j}\right)
\end{align*}

Elasticity given by
\begin{align*}
E_{jj} & =-\alpha s_{j}\left(1-s_{j}\right)\frac{p_{j}}{s_{j}}\\
 & =-\alpha\left(1-s_{j}^{2}\right)p_{j}
\end{align*}


\paragraph{Cross-price elasticity.
\begin{align*}
E_{jk} & =\frac{\partial s_{j}}{\partial p_{k}}\frac{p_{k}}{s_{j}}\protect\\
\end{align*}
}

Start with the own-price elasticity.

\paragraph{
\begin{align*}
\frac{\partial s_{j}}{\partial p_{k}} & =\frac{\partial}{\partial p_{k}}\left[\frac{e^{-\alpha p_{j}}}{\sum_{l=0}^{J}e^{-\alpha p_{l}}}\right]\protect\\
 & =\left(-1\right)e^{-\alpha p_{j}}\frac{1}{\left[\sum_{l=0}^{J}e^{-\alpha p_{l}}\right]^{2}}e^{-\alpha p_{k}}\left(-\alpha\right)\protect\\
 & =\alpha s_{j}s_{k}
\end{align*}
}

Elasticities.
\begin{align*}
E_{jk} & =\alpha s_{j}s_{k}\frac{p_{k}}{s_{j}}\\
 & =\alpha s_{k}p_{k}
\end{align*}

The cross-price elasticity is not plausible. Here the cross-price
elasticity depends \emph{only} on the other good's expenditure share.
It isn't super sensible because good may be closer or further subsitutes
in a way that doesn't depend on market share.

\begin{problem}{4.c}
(c) Assume that $\epsilon_{i j}=\beta_{i} x_{j}$ where $x_{j}$, represents a non-random product characteristic that consumers value, and $\beta_{i}$ represents an idiosyncratic taste shock for that same characteristic. Moreover, assume that $x_{j}>0, x_{0}=0$.
\end{problem}

\begin{problem}{4.c.i}
Assume that $\beta_{i} \equiv \beta$ for all $i$. Derive product $j$ market share, $s_{j}$. Interpret your results.
\end{problem}

\paragraph{Solution.}

By assuming that $\beta_{i}\equiv\beta$ is constant and equal for
all $i$ utility is deterministic. Consumer $i$ chooses product $j$
if for all $k$
\[
\alpha(y_{i}-p_{j})+\beta x_{j}>\alpha(y_{i}-p_{k})+\beta x_{k}
\]
. Cancelling out income again, we get that the market share for product
$j=0,1,\ldots,J$ is
\[
s(j)=\begin{cases}
1 & \text{if }\beta x_{j}-\alpha p_{j}=\max_{l}\{\beta x_{l}-\alpha p_{l}\}\\
0 & o/w
\end{cases}
\]

Now all consumers are identical so product $j$ either dominates the
market if it is the best product and has zero market share if it is
not the best.

\begin{problem}{4.c.ii}
Assume that $\beta_{i}$ are i.i.d Uniform $[0, \bar{\beta}]$ with $\bar{\beta}$ sufficiently large. Derive product $j$ market share, $s_{j}$ and compute own and cross-price elasticities. Are the latter reasonable? Explain and compare with your findings in points (b) above. (For simplicity assume that $\frac{p_{i}-p_{j}}{x_{i}-x_{j}} \geq \frac{p_{j}-p_{k}}{x_{j}-x_{k}}$ whenever $x_{i} \geq x_{j} \geq x_{k}$ )
\end{problem}

Consider two goods. Ignoring ties, the probability that consumer $i$
will prefer product $j$ over product $k$ is 

\begin{align*}
\Pr\{\text{i prefers j over k}\} & =\Pr\{\beta_{i}x_{j}-\alpha p_{j}>\beta_{i}x_{k}-\alpha p_{k}\}\\
 & =\Pr\left\{ -\alpha\left(p_{j}-p_{k}\right)>\beta_{i}\left(x_{k}-x_{j}\right)\right\} \\
 & =\Pr\left\{ \alpha\frac{p_{j}-p_{k}}{x_{j}-x_{k}}>\beta_{i}\right\} \\
 & =\left\{ \begin{array}{ll}
F_{U}\left(\alpha\frac{p_{j}-p_{k}}{x_{j}-x_{k}}\right), & \text{if }x_{j}>x_{k}\\
1-F_{U}\left(\alpha\frac{p_{j}-p_{k}}{x_{j}-x_{k}}\right), & \text{if }x_{j}<x_{k}
\end{array}\right.
\end{align*}
 where $F_{U}$ is the CDF of the uniform distribution over $[0,\overline{\beta}]$.
Now consider $J$ goods and ignore ties. We have the probability of
consumer $i$ choosing $j$ is
\[
\Pr\{\text{i prefers j}\}=\Pr\left\{ \beta_{i}>\alpha\frac{p_{j}-p_{l}}{x_{j}-x_{l}}\text{ }\forall l:x_{j}>x_{l}\text{ and }\beta_{i}<\alpha\frac{p_{j}-p_{m}}{x_{j}-x_{m}}\text{ }\forall m:x_{j}<x_{m}\right\} 
\]
 By assumption, if $x_{j}>x_{l}>x_{k}$, then $\frac{p_{j}-p_{l}}{x_{j}-x_{l}}>\frac{p_{j}-p_{k}}{x_{j}-x_{k}}$.
So for good $j$ that is not the highest or lowest-quality, we only
need to compare the goods just above and just below good $j$ in quality.
WLOG assume goods are ordered by quality: $x_{0}<x_{1}<\ldots<x_{J}$.
For an intermediate good, the above choice probability becomes:
\begin{align*}
\Pr\{\text{i prefers j}\} & =\Pr\left\{ \beta_{i}\in\left(\alpha\frac{p_{j}-p_{j-1}}{x_{j}-x_{j-1}},\alpha\frac{p_{j}-p_{j+1}}{x_{j}-x_{j+1}}\right)\right\} \\
 & =\frac{\alpha}{\overline{\beta}}\left(\frac{p_{j}-p_{j-1}}{x_{j}-x_{j-1}}-\frac{p_{j}-p_{j+1}}{x_{j}-x_{j+1}}\right)
\end{align*}
 Incorporating the goods with the highest and lowest quality into
this, we get 
\[
\Pr\{\text{i prefers j}\}=\left\{ \begin{array}{ll}
1-\frac{\alpha}{\overline{\beta}}\frac{p_{j}-p_{j-1}}{x_{j}-x_{j-1}}, & \text{if }x_{j}=\max_{l}x_{l}\\
\frac{\alpha}{\overline{\beta}}\frac{p_{j}-p_{j+1}}{x_{j}-x_{j+1}}, & \text{if }x_{j}=\min_{l}x_{l}\\
\frac{\alpha}{\overline{\beta}}\left(\frac{p_{j}-p_{j-1}}{x_{j}-x_{j-1}}-\frac{p_{j}-p_{j+1}}{x_{j}-x_{j+1}}\right), & \text{otherwise}
\end{array}\right.
\]
 Since there are a large number of consumers, this choice probability
will approximate the market share $s(j)$. Own-price elasticity will
be 

\[
\begin{align*}\varepsilon_{jj} & =\frac{\partial s(j)}{\partial p_{j}}\frac{p_{j}}{s(j)}\\
 & =\left\{ \begin{array}{ll}
-\frac{p_{j}}{s(j)}\frac{\alpha}{\overline{\beta}}\frac{1}{x_{j}-x_{j-1}}, & \text{if }x_{j}=\max_{l}x_{l}\\
\frac{p_{j}}{s(j)}\frac{\alpha}{\overline{\beta}}\frac{1}{x_{j}-x_{j+1}}, & \text{if }x_{j}=\min_{l}x_{l}\\
\frac{p_{j}}{s(j)}\frac{\alpha}{\overline{\beta}}\left(\frac{1}{x_{j}-x_{j-1}}-\frac{1}{x_{j}-x_{j+1}}\right), & \text{otherwise}
\end{array}\right.
\end{align*}
\]

Cross-price elasticities will be zero for all goods that are not immediately
adjacent in quality. For good $j+1$, which is the good with the next
highest quality after good $j$, we have 

\[
\varepsilon_{j,j+1}=\frac{\partial s(j)}{\partial p_{j+1}}\frac{p_{j+1}}{s(j)}=\left\{ \begin{array}{ll}
0, & \text{if }x_{j}=\max_{l}x_{l}\\
-\frac{p_{j+1}}{s(j)}\frac{\alpha}{\overline{\beta}}\frac{1}{x_{j}-x_{j+1}}, & \text{if }x_{j}=\min_{l}x_{l}\\
\frac{p_{j+1}}{s(j)}\frac{\alpha}{\overline{\beta}}\frac{1}{x_{j}-x_{j+1}}, & \text{otherwise}
\end{array}\right.
\]

For good $j-1$, which is the good with the next lowest quality after
good $j$, we have 

\[
\begin{align*}\varepsilon_{j,j-1} & =\frac{\partial s(j)}{\partial p_{j-1}}\frac{p_{j-1}}{s(j)}\\
 & =\left\{ \begin{array}{ll}
\frac{p_{j-1}}{s(j)}\frac{\alpha}{\overline{\beta}}\frac{1}{x_{j}-x_{j-1}}, & \text{if }x_{j}=\max_{l}x_{l}\\
0, & \text{if }x_{j}=\min_{l}x_{l}\\
-\frac{p_{j-1}}{s(j)}\frac{\alpha}{\overline{\beta}}\frac{1}{x_{j}-x_{j-1}}, & \text{otherwise}
\end{array}\right.
\end{align*}
\]

This is a more complicated substitution pattern than before. Now,
only one or two goods can induce any substitution effects. This is
no longer IIA.

\begin{problem}{4.d}
Assume that $\epsilon_{i j}=\beta_{i} x_{j}+\nu_{i j}$ where $x_{j}$ represents a non-random product characteristic, $\beta_{i}$ represents an idiosyncratic taste shock for that same characteristic and $v_{i j}$ are i.i.d $\operatorname{T1EV}(0,1)$. Moreover, assume that $\beta_{i}$ are i.i.d with generic c.d.f $F(\cdot)$. Derive product j's market share and compute own and cross-price elasticities. Explain and compare with your findings in point (b) above.
\end{problem}

The T1EV(0,1) error $v_{ij}$ gives us a logistic choice probability,
but now we integrate over random coefficient $\beta_{i}$. Given a
large number of buyers, we approximate the market share for product
\$j\$ with individual choice probability: 

\[
s(j)\approx s_{i}(j)=\int_{\beta_{i}}\frac{\exp(-\alpha p_{j}+\beta_{i}x_{j})}{\sum_{l=0}^{J}\exp(-\alpha p_{l}+\beta_{i}x_{l})}dF(\beta_{i})
\]
 We have the following partial derivatives of market share with respect
to own and cross prices:

\begin{align*}
\frac{\partial s(j)}{\partial p_{j}} & =\int_{\beta_{i}}\bigg(\frac{\exp(-\alpha p_{j}+\beta_{i}x_{j})}{\sum_{l=0}^{J}\exp(-\alpha p_{l}+\beta_{i}x_{l})}(-\alpha)-\frac{\exp(-\alpha p_{j}+\beta_{i}x_{j})}{(\sum_{l=0}^{J}\exp(-\alpha p_{l}+\beta_{i}x_{l}))^{2}}(\exp(-\alpha p_{j}+\beta_{i}x_{j}))(-\alpha)\bigg)dF(\beta_{i})\\
 & =-\alpha\int_{\beta_{i}}\left(\frac{\exp(-\alpha p_{j}+\beta_{i}x_{j})}{\sum_{l=0}^{J}\exp(-\alpha p_{l}+\beta_{i}x_{l})}\right)\left(1-\frac{\exp(-\alpha p_{j}+\beta_{i}x_{j})}{\sum_{l=0}^{J}\exp(-\alpha p_{l}+\beta_{i}x_{l})}\right)dF(\beta_{i})\\
 & =-\alpha\int_{\beta_{i}}s_{i}(j|\beta_{i})(1-s_{i}(j|\beta_{i}))dF(\beta_{i})\frac{\partial s(j)}{\partial p_{k}}\\
 & =\int_{\beta_{i}}\bigg(-\frac{\exp(-\alpha p_{j}+\beta_{i}x_{j})}{(\sum_{l=0}^{J}\exp(-\alpha p_{l}+\beta_{i}x_{l}))^{2}}(\exp(-\alpha p_{k}+\beta_{i}x_{k}))(-\alpha)\bigg)dF(\beta_{i})\\
 & =\alpha\int_{\beta_{i}}\left(\frac{\exp(-\alpha p_{j}+\beta_{i}x_{j})}{\sum_{l=0}^{J}\exp(-\alpha p_{l}+\beta_{i}x_{l})}\right)\left(\frac{\exp(-\alpha p_{k}+\beta_{i}x_{k})}{\sum_{l=0}^{J}\exp(-\alpha p_{l}+\beta_{i}x_{l})}\right)dF(\beta_{i})\\
 & =\alpha\int_{\beta_{i}}s_{i}(j|\beta_{i})s_{i}(k|\beta_{i})dF(\beta_{i})
\end{align*}
 where $s_{i}(j|\beta_{i})$ is the individual choice probability
given a fixed $\beta_{i}$ for that person, which reduces to the logit
model. We use these partial derivatives to derive own-price and cross-price
elasticities: 

\[
\begin{align*}\varepsilon_{jj} & =\frac{\partial s(j)}{\partial p_{j}}\frac{p_{j}}{s(j)}\\
 & =-\frac{\alpha p_{j}}{s(j)}\int_{\beta_{i}}s_{i}(j|\beta_{i})(1-s_{i}(j|\beta_{i}))dF(\beta_{i})\\
\varepsilon_{jk} & =\frac{\partial s(j)}{\partial p_{k}}\frac{p_{k}}{s(j)}\\
 & =\frac{\alpha p_{k}}{s(j)}\int_{\beta_{i}}s_{i}(j|\beta_{i})s_{i}(k|\beta_{i})dF(\beta_{i})
\end{align*}
\]

If the $\beta_{i}$ were constant across all individuals, then we
would get the same result as we do in part (b). These elasticities
are much more flexible, however, as a result of the potential heterogeneity
in $\beta_{i}$.

\begin{problem}{4.e}
Assume, as in point (a) above, that $\epsilon_{i j}$ are i.i.d T1EV(0,1). Moreover, suppose we want to measure welfare at given prices $\left(p_{1}, \ldots, p_{J}\right)$ as
$$ W \equiv \mathbb{E}\left[\max _{j=0, \ldots, J} u_{i j}\right] $$
i. Rewrite $W$ as a function of the market share of the outside option $s_{0} \cdot{ }^{1}$
ii. Suppose that a new product $J+1$ is introduced in the market. What happens to $W$ ? Interpret your results.
\end{problem}

\begin{problem}{i}
Rewrite $W$ as a function of the market share of the outside option $s_{0} \cdot{ }^{1}$
\end{problem}

Recall that welfare can be written as 

\begin{align*}
W & :=\E\left[\max_{j}\{\alpha y_{i}-\alpha p_{j}+\epsilon_{ij}\}\right]\\
 & =\alpha y_{i}+\E\left[\max_{j}\{-\alpha p_{j}+\epsilon_{ij}\}\right]
\end{align*}

Note that since $\epsilon_{ij}\sim T1EV\left(0,1\right)$ we can write
$-\alpha p_{j}+\epsilon_{ij}\overset{iid}{\sim}T1EV\left(-\alpha p_{j},1\right)$.
From part (a) we know that the maximum of a Gumbel is Gumbel.

\[
\max_{j}\{-\alpha p_{j}+\epsilon_{ij}\}\sim\text{T1EV}\left(\log(\sum_{j=0}^{J}\exp(-\alpha p_{j})),1\right)
\]

Taking expectations:
\begin{align*}
\mathbb{E}\left[\max_{j}\{-\alpha p_{j}+\epsilon_{ij}\}\right] & =\log(\sum_{j=0}^{J}\exp(-\alpha p_{j}))+\underbrace{\gamma}_{\text{Euler-Mascheroni constant}}\\
\end{align*}

So expected welfare comes to 
\begin{align*}
W & :=\E\left[\max_{j}\{\alpha y_{i}-\alpha p_{j}+\epsilon_{ij}\}\right]\\
 & =\alpha y_{i}+\log(\sum_{j=0}^{J}\exp(-\alpha p_{j}))+\gamma
\end{align*}

Recall that the $s_{j}=\frac{e^{-\alpha p_{j}}}{\sum_{k=0}^{J}e^{-\alpha p_{k}}}$.
The outside option $j=0$ has a price normalized to zero $p_{0}=0$.
\begin{align*}
s_{0} & =\frac{e^{-\alpha0}}{\sum_{k=0}^{J}e^{-\alpha p_{k}}}\\
 & =\frac{1}{\sum_{k=0}^{J}e^{-\alpha p_{k}}}
\end{align*}

Plugging into the welfare equation
\begin{align*}
W & =\alpha y_{i}+\log(\sum_{j=0}^{J}\exp(-\alpha p_{j}))+\gamma\\
 & =\alpha y_{i}+\log(\frac{1}{s_{0}})+\gamma
\end{align*}
 Welfare decreases as folks leave the market and purchase the outside
good $s_{0}$.

\begin{problem}{ii}
Suppose that a new product $J+1$ is introduced in the market. What happens to $W$? Interpret your results.
\end{problem}

Welfare weakly increases if a new product $J+1$ is introduced.

\begin{align*}
W^{\text{new}} & =\alpha y_{i}+\log(\underbrace{\exp(-\alpha p_{J+1})}_{\geq0}+\sum_{j=0}^{J}\exp(-\alpha p_{j}))+\gamma
\end{align*}

If $p_{J+1}<\infty$ then welfare increases. The reason for this is
that welfare is increasing in the size of the market (decreasing in
the outside option). Increasing the number of goods in this setting
means increasing the market size. The new product $J+1$ is stealing
market share equally from all other goods (IIA issue of iid errors)
\emph{including }the outside option. So total market size grows so
welfare increases.

\section{Estimation - homogeneous coefficients with demographics }

The file ps1\_ex2.csv contains data on $I=4000$ individual choices
among $J=30$ products and an outside good denoted by $j=31$. Assume
individual $i$ 's indirect utility of consuming product $j$ is

\[
\begin{aligned} & u_{ij}=x_{j}^{\prime}\beta+\xi_{j}+d_{i}^{\prime}\Gamma x_{j}+\epsilon_{ij}\quad\text{ for }j=1,\ldots,30\\
 & u_{ij}=\epsilon_{ij}\quad\text{ for }j=31
\end{aligned}
\]

where $x_{j}$ is a $K$ dimensional vector of product characteristics,
$\xi_{j}$ is an unobserved product characteristic, $d_{i}$ is an
$L$ dimensional vector of individual observable demographics and
$\epsilon_{ij}$ is an i.i.d $T1EV(0,1)$ taste shock. Our goal is
to estimate the coefficients on product characteristics enclosed in
the $K$ dimensional vector $\beta$ and all interaction coefficients
between product characteristics and individual demographics enclosed
in the $(L\times K)$ matrix $\Gamma$.

\begin{problem}{1}
What are the coefficients in $\beta$ capturing? What about the coefficients in $\Gamma$?
\end{problem}

$\beta$ captures how much observable characteristics affect consumer
utility. $\Gamma_{ij}$ is the coefficient on $d_{i}\next x_{j}$
and captures how people with demographics $d_{i}$ (e.g. parents)
value characteristic $x_{j}$ (e.g. car size) relative to the average
consumer. 

\begin{problem}{2}
Denote by $\delta_{j}\equiv x_{j}^{\prime}\beta+\xi_{j}$ the mean utility of product $j$  (where clearly $\delta_{31}=0$ ). Write down the log-likelihood of the data as a function of the parameters $(\delta,\Gamma)$ where $\delta=\left(\delta_{j}\right)_{j=1}^{J}$.
\end{problem}

Write indirect utility as 
\[
u_{ij}=\delta_{j}+d_{i}^{\prime}\Gamma x_{j}+\epsilon_{ij}
\]

We know from previous results that the distribution assumption on
$\epsilon_{ij}$ yields 
\begin{align*}
Pr\left(i\rightarrow j\right) & =\frac{exp\text{\ensuremath{\left(\delta_{j}+d_{i}^{\prime}\Gamma x_{j}\right)}}}{\sum_{k\neq j}exp\left(\delta_{j}+d_{i}^{\prime}\Gamma x_{j}\right)}\\
\end{align*}

So the likelihood and log-likelihood function can be written as 
\begin{align*}
\mathcal{L}\left(y_{ij},d_{i},x_{j}|\delta,\Gamma\right) & =\prod_{i=1}^{I}\prod_{j=1}^{J+1}\left[\frac{exp\text{\ensuremath{\left(\delta_{j}+d_{i}^{\prime}\Gamma x_{j}\right)}}}{\sum_{k\neq j}exp\left(\delta_{j}+d_{i}^{\prime}\Gamma x_{j}\right)}\right]^{y_{ij}}\\
\ell\left(y_{ij},d_{i},x_{j}|\delta,\Gamma\right) & =\sum_{i=1}^{I}\sum_{j=1}^{J+1}y_{ij}\left[\delta_{j}+d_{i}^{\prime}\Gamma x_{j}-\log\sum_{k\neq j}exp\left(\delta_{j}+d_{i}^{\prime}\Gamma x_{j}\right)\right]
\end{align*}

\begin{problem}{3}
Derive the FOC of the log-likelihood with respect to $\delta_{j}$. Interpret your result.
\end{problem}

Taking derivative with respect to $\delta_{j}$

\begin{align*}
\frac{\partial\ell\left(y_{ij},d_{i},x_{j}|\delta,\Gamma\right)}{\partial\delta_{j}} & =\sum_{i=1}^{I}y_{ij}-\sum_{i=1}^{I}\sum_{k\neq j}y_{ik}\frac{exp\left(\delta_{j}+d_{i}^{\prime}\Gamma x_{j}\right)}{\sum_{l\neq k}exp\left(\delta_{l}+d_{i}^{\prime}\Gamma x_{l}\right)}
\end{align*}

MLE sets this FOC equal to zero implying
\begin{align*}
\sum_{i=1}^{I}y_{ij} & =\sum_{i=1}^{I}\sum_{k\neq j}y_{ik}\frac{exp\left(\delta_{j}+d_{i}^{\prime}\Gamma x_{j}\right)}{\sum_{l\neq k}exp\left(\delta_{l}+d_{i}^{\prime}\Gamma x_{l}\right)}\\
 & =exp\left(\delta_{j}+d_{i}^{\prime}\Gamma x_{j}\right)\sum_{i=1}^{I}\sum_{k\neq j}y_{ik}\times\frac{1}{\sum_{l\neq k}exp\left(\delta_{l}+d_{i}^{\prime}\Gamma x_{l}\right)}
\end{align*}

On the LHS we have the total number of individuals who purchase good
$j$. On the RHS we product of two terms. The first $exp\left(\delta_{j}+d_{i}^{\prime}\Gamma x_{j}\right)$
is increasing in $\delta_{j}$. So assuming $d_{i}\Gamma x_{j}$ fixed
and sales and characteristics of other goods are fixed, MLE spits
out a larger $\delta_{j}$ if sales of $j$ are larger. 

\begin{problem}{4}
Derive the FOC of the log-likelihood with respect to $\Gamma$. Interpret your results.
\end{problem}

Now with respect to $\Gamma$

\begin{align*}
\frac{\partial\ell\left(y_{ij},d_{i},x_{j}|\delta,\Gamma\right)}{\partial\Gamma} & =\sum_{i=1}^{I}\sum_{j=1}^{J+1}y_{ij}d_{i}^{\prime}x_{j}-\sum_{i=1}^{I}\sum_{j=1}^{J+1}y_{ij}\left[\frac{1}{\sum_{k\neq j}exp\left(\delta_{j}+d_{i}^{\prime}\Gamma x_{j}\right)}\times d_{i}^{\prime}\sum_{k\neq j}exp\left(\delta_{j}+d_{i}^{\prime}\Gamma x_{j}\right)x_{j}\right]\\
 & =\sum_{i=1}^{I}\sum_{j=1}^{J+1}y_{ij}d_{i}^{\prime}x_{j}-\sum_{i=1}^{I}\sum_{j=1}^{J+1}y_{ij}d_{i}^{\prime}x_{j}\\
 & =0
\end{align*}

Shrug emoji. Definitely an error here in my derivative. 

s

s

s

s

s

s

s

s

s

s

s

s

s

3. 

5. Obtain MLE estimates for \$(\textbackslash delta, \textbackslash Gamma)\$.

6. Exploiting your MLE estimates for the product specific intercepts
\$\textbackslash hat\{\textbackslash delta\}\_\{M L E\}\$ you now
want to estimate \$\textbackslash beta\$. Propose a moment condition
that (if true) would allow you to consistently estimate \$\textbackslash beta\$.

7. Use the proposed moment condition to obtain an estimate of \$\textbackslash beta\$.

\textbackslash section\{Estimation - costs, conduct, and counterfactuals\}

The file ps1\_ex3.csv contains aggregate data on a large number \$T=1000\$
of markets in which \$J=6\$ products compete between each other together
with an outside good \$j=0\$. The utility of consumer \$i\$ is given
by:

\$\$ \textbackslash begin\{aligned\} \& u\_\{i j t\}=-\textbackslash alpha
p\_\{j t\}+\textbackslash beta x\_\{j t\}+\textbackslash xi\_\{j
t\}+\textbackslash epsilon\_\{i j t\} \textbackslash quad j=1, \textbackslash ldots,
6 \textbackslash\textbackslash{} \& u\_\{i 0 t\}=\textbackslash epsilon\_\{i
0 t\} \textbackslash end\{aligned\} \$\$

where \$p\_\{j t\}\$ is the price of product \$j\$ in market \$t,
x\_\{j t\}\$ is an observed product characteristic, \$\textbackslash xi\_\{j
t\}\$ is an unobserved product characteristic and \$\textbackslash epsilon\_\{i
j t\}\$ is i.i.d \$\textbackslash mathrm\{T\} 1 \textbackslash mathrm\{EV\}(0,1)\$.
Our goal is to to estimate demand parameters \$(\textbackslash alpha,
\textbackslash beta)\$ and perform some counterfactual exercise.

1. Assuming that the variable \$z\$ in the dataset is a valid instrument
for prices, write down the

\$\{ \}\textasciicircum\{2\}\$ Note that in the data \$x\_\{31\}=0\$.
moment condition that allows you to consistently estimate \$(\textbackslash alpha,
\textbackslash beta)\$ and obtain an estimate for both parameters.

2. For each market, compute own and cross-product elasticities. Average
your results across markets and present them in a \$J \textbackslash times
J\$ table whose \$(i, j)\$ element contains the (average) elasticity
of product \$i\$ with respect to an increase in the price of product
\$j\$. What do you notice?

3. Using your demand estimates, for each product in each market recover
the marginal cost \$c\_\{j t\}\$ implied by Nash-Bertrand competition.
For simplicity, you can assume that in each market each product is
produced by a different firm (i.e., there is no multi-products firms).
Report the average (across markets) marginal cost for each product.
Could differences in marginal costs explain the differences in the
average (across markets) market shares and prices that you observe
in the data?

4. Suppose that product \$j=1\$ exits the market. Assuming that marginal
costs and product characteristics for the other products remain unchanged,
use your estimated marginal costs and demand parameters to simulate
counterfactual prices and market shares in each market. Report the
resulting average prices and shares.

5. Finally, for each market compute the change in firms profits and
in consumer welfare following the exit of firm \$j=1\$. Report the
average changes across markets. Who wins and who loses?

\textbackslash section\{Estimation - BLP\}

The file ps1\_ex4.csv contains aggregate data on \$T=100\$ markets
in which \$J=6\$ products compete between each other together with
an outside good \$j=0\$. The utility of consumer \$i\$ is given by:

\$\$ \textbackslash begin\{aligned\} \& u\_\{i j t\}=\textbackslash tilde\{x\}\_\{j
t\}\textasciicircum\{\textbackslash prime\} \textbackslash beta+\textbackslash xi\_\{j
t\}+\textbackslash tilde\{x\}\_\{j t\}\textasciicircum\{\textbackslash prime\}
\textbackslash Gamma v\_\{i\}+\textbackslash epsilon\_\{i j t\}
\textbackslash quad j=1, \textbackslash ldots, 6 \textbackslash\textbackslash{}
\& u\_\{i 0 t\}=\textbackslash epsilon\_\{i 0 t\} \textbackslash end\{aligned\}
\$\$

where \$x\_\{j t\}\$ is a vector of observed product characteristics
including the price, \$\textbackslash xi\_\{j t\}\$ is an unobserved
product characteristic, \$v\_\{i\}\$ is a vector of unobserved taste
shocks for the product characteristics and \$\textbackslash epsilon\_\{i
j t\}\$ is i.i.d \$\textbackslash operatorname\{T1EV\}(0,1)\$. Our
goal is to to estimate demand parameters \$(\textbackslash beta,
\textbackslash Gamma)\$ using the BLP algorithm. As you can see from
the data there are only two characteristics \$\textbackslash tilde\{x\}\_\{j
t\}=\textbackslash left(p\_\{j t\}, x\_\{j t\}\textbackslash right)\$,
namely prices and an observed measure of product quality. Moreover,
there are several valid instruments \$z\_\{j t\}\$ that you will use
to construct moments to estimate \$(\textbackslash alpha, \textbackslash Gamma)\$.
Finally, you can assume that \$\textbackslash Gamma\$ is lower triangular
e.g.,

\$\$ \textbackslash Gamma=\textbackslash left{[}\textbackslash begin\{array\}\{cc\}
\textbackslash gamma\_\{11\} \& 0 \textbackslash\textbackslash{}
\textbackslash gamma\_\{21\} \& \textbackslash gamma\_\{22\} \textbackslash end\{array\}\textbackslash right{]}
\$\$

such that \$\textbackslash Gamma \textbackslash Gamma\textasciicircum\{\textbackslash prime\}=\textbackslash Omega\$
is a positive definite matrix and that \$v\_\{i\}\$ is a 2 dimensional
vector of i.i.d random taste shocks. 1. Assume that \$v\_\{i\} \textbackslash sim
N(0, I)\$ so that \$\textbackslash Gamma v\_\{i\} \textbackslash sim
N(0, \textbackslash Omega)\$. Implement the BLP routine to estimate
\$(\textbackslash beta, \textbackslash Gamma)\$. You may want to
write multiple functions, including but not limited to: a share prediction
function, a share inversion function, an implicit function \$\textbackslash xi(\textbackslash beta,
\textbackslash Gamma)\$, and an objective function to be minimized.

2. For each market, compute cross and own product elasticities. Average
your results across markets and present them in a \$J \textbackslash times
J\$ table whose \$(i, j)\$ element contains the (average) elasticity
of product \$i\$ with respect to an increase in the price of product
\$j\$. What's the main difference when compared with the table of
elasticities you found in \$3.2\$ ?

3. Look at the average (across markets) prices, shares and observed
quality of the products you observe in the data. Based on your estimated
\$\textbackslash Gamma\$, what do you think could be driving differences
in prices and market shares?

4. (Optional) compare your results with PyBLP (this does not need
to be done in Python - you can call PyBLP from R, Matlab, and Julia).

s

s

s

s

s

s

s

s

ss

s

s

s
\end{document}
